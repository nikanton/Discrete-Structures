\begin{task}{322}
Пусть $n$ — произвольное натуральное число. Пусть $S_1, \dots, S_{n^{2017}}$ — произвольные $n$‑элементные множества. Докажите, что при всех достаточно больших значениях $n$ можно покрасить элементы в красный и синий цвета, так, чтобы в каждом из множеств $S_i$ нашёлся хотя бы один красный и хотя бы один синий элемент.
\end{task}

\begin{solution}
Покрасим элементы множеств независимо друг от друга в два цвета (красный и синий) с одинаковой вероятностью. Тогда вероятность обоих событий равна $\frac{1}{2}$. По условию нужно показать, что
\begin{equation*}
    P(\text{в каждом }S_i\text{ есть элементы обоих цветов})>0.
\end{equation*}
Это эквивалентно доказательству того, что
\begin{equation}\label{P(A)}
    P(A)\equiv P(\exists S_i\text{, элементы которого покрашены в один цвет})<1.
\end{equation}
Отдельно рассмотрим следующий случай:
\begin{align}\label{P(A_i)}
    P(A_i)
    &\equiv P(S_i\text{ покрашен в один цвет})=\nonumber \\
      &=P(S_i\text{ покрашен полностью в красный})+\nonumber \\
      &\qquad+P(S_i\text{ покрашен полностью в синий})=\nonumber\\
    &=\left(\frac{1}{2}\right)^n+\left(\frac{1}{2}\right)^n=\left(\frac{1}{2}\right)^{n-1}.
\end{align}
Покажем, что \eqref{P(A)} выполняется, воспользовавшись результатом \eqref{P(A_i)}.
\begin{equation}\label{result}
    P(A)=P\left(\bigcup_{i = 1}^{n^{2017}}A_i\right)\leq\sum_{i = 1}^{n^{2017}}P(A_i)=\frac{n^{2017}}{2^{n-1}}
\end{equation}
Выполним предельный переход в \eqref{result}.
\begin{equation*}
    P(A)=2^{2017\log{n}+1-n}\xrightarrow{n\rightarrow \infty} 0
\end{equation*}
Таким образом, мы доказали \eqref{P(A)} и получили, что при достаточно больших значениях $n$ существует раскраска, где в каждом $S_i$ найдутся элементы обоих цветов. 
\end{solution}
