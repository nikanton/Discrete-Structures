\subsection{Формулы}
Можно писать обычные формулы в тексте, наподобие $4xy = (x+y)^2+(x-y)\cdot(x-y)$. Можно делать и выключные формулы: \[-1 = e^{i\cdot \pi}.\]

Можно даже делать формулы с номером с помощью \verb"\begin{equation}":
\begin{equation}
\label{aMasterEquation}
E=mc^2,
\end{equation}
и потом ссылаться на них так:~\eqref{aMasterEquation}.


\subsection{Перекрёстные ссылки}
Можно, кстати, ссылаться на параграфы, например, можно сослаться на параграф~\ref{Algo}. Можно сослаться и на страницу~\pageref{Algo}, на которой находится параграф~\ref{Algo}.

\newpage % Разрыв страницы. Его можно сделать и командой \pagebreak

\subsection{Списки}
Есть два типа списков:
\begin{enumerate}
\item нумерованные списки,
    \begin{enumerate}
    \item причём их можно вкладывать друг в друга,
    \item вот такие дела,
    \item\label{aListSubitem} красота,
    \end{enumerate}
\item и ненумерованные списки:
    \begin{itemize}
    \item ученье
    \item свет
    \end{itemize}
\end{enumerate}

Связка команд \verb"\label" и \verb"\ref" работает и тут: например, можно сослаться на пункт \ref{aListSubitem}.


\subsection{Описания алгоритмов}\label{Algo}
Код на Питоне:
\begin{python}
def maximum( x, y ):
    #something strange happens here
    if x > y:
        print( "x was greater" )
        return x
    else:
        return y
\end{python}

Псевдокод:
\begin{algorithm}\caption{Поиск в глубину}
\DontPrintSemicolon
\KwData{граф $G=(V,E)$}
\KwResult{список вершин $L$ в порядке DFS}
\Begin{
$S \longleftarrow V$\;
$L \longleftarrow \emptyset$\;
\While{$S \neq \emptyset$}{
    \While{$|S \cap ImSucc(x)| \neq |S|$}{
        \If{$NbPredInMin(x) = 0$ {\bf and} $NbPredNotInMin(x) = 0$}{
            $AppendToMin(x)$
        }
        \For{$z \in ImPred(y) \cap Min$}{
            remove the arc $zy$ from $V$\;
            $NbSuccInS(z) \longleftarrow NbSuccInS(z) - 1$\;
            move $z$ in $T$ to the list preceding its present list\;
            \{i.e. If $z \in T[k]$, move $z$ from $T[k]$ to $T[k-1]$\}\;
        }
        $NbPredInMin(y) \longleftarrow 0$\;
    }
}
}
\end{algorithm}


По-видимому, Ферма всё же не обладал доказательством того, что уравнение $x^n+y^n=z^n$ разрешимо в натуральных числах только при $n=2$. К счастью, сегодня такое доказательство известно. Сегодня особенно удобно оформлять математические тексты, даже такие длинные, как доказательство теоремы Ферма, благодаря \LaTeX. Если потребуются особые шрифты, они есть:
\[ n\in\mathbb{N},\,a\in\mathcal{A},\, p\in\mathrm{P} \]