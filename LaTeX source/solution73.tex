\begin{task}{73}
Квадратная таблица $(2n+1)\times(2n+1)$, где $n\in\mathbb{N}$, заполнена числами от $1$ до $2n+1$ так, что в каждой строке и в каждом столбце представлены все эти числа. Докажите, что если это расположение симметрично относительно диагонали таблицы, то на этой диагонали тоже представлены все эти числа. В решении используйте принцип Дирихле, обязательно явно указав все детали применения (если пользуетесь «клеточно-кроликовой» терминологией, то что в Вашей задаче играет роль «клеток» и что — роль «кроликов»; если более строгой формулировкой, то как общая формулировка «ложится» на конкретику Вашей задачи).
\end{task}

\begin{solution}
Пусть какое-то число на диагонали встречается дважды, значит на диагонали остается $2n-1$ пустая клетка. Всего различных неиспользованных чисел остаётся $2n$ (это будут кролики). По принципу Дирихле найдётся число, которое не будет выписано на диагонали. Обозначим такое число за $k$. Если число с аналогичным значением $k$ встречается в клетке с координатами $(x, y)$, то встречается и в клетке $(y, x)$, при этом $y\neq x$. Получается, все числа со значением $k$ разбиты на не повторяющиеся пары, но чисел $k$ --- $2n+1$ штук (нечётное количество). Приходим к противоречию. Получается, что каждое число встречается на диагонали по одному разу. 
\end{solution}