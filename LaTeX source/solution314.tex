\begin{task}{314}
\begin{enumerate}
\item Какое количество ребер может быть в пятивершинном порожденном подграфе полного десятивершинного графа?
\item Какое максимальное количество ребер может быть в остовном подграфе полного десятивершинного графа?
\item Как могла бы выглядеть «теорема о рукопожатиях» для гиперграфов?
\item В графе $G$ имеется $3k$ изолированных вершин, $2m$ висячих вершин, $n$ проходных вершин, а других вершин в $G$ нет. Чему равняется $\|G\|$? Ответ запишите в виде формулы от $m$, $n$, $k$.
\end{enumerate}
\end{task}

\begin{solution}
\begin{enumerate}
\item В изначальном графе между любой парой вершин было ребро, так как это $K_{10}$. Выберем любые 5 вершин в нём. Между любыми из них будет ребро. По определению порождённого подграфа имеем, что любой порождённый пятивершинный граф из $K_{10}$ будет изоморфен $K_5$. Получается, что его число рёбер $\frac{5(5-1)}{2}=10$.
\item Остовный подграф содержит все вершины исходного графа и часть из его рёбер. Очевидно, максимальное значение будет достигаться, когда в остовный подграф войдут все рёбра, $\frac{10(10-1)}{2}=45$.
\item Было решено.
\item По теореме о рукопожатиях удвоенное число рёбер равно сумме степеней вершин. Получаем ответ: $\frac{0 \cdot 3k+1 \cdot 2m+2 \cdot k}{2}=m+k$
\end{enumerate}
\end{solution}