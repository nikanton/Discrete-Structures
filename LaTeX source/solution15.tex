\begin{task}{15}
На потоке второго курса ПМФ ФИВТ три группы. Преподаватели узнали, что каждый второкурсник ПМФ всегда даёт списывать ровно одному своему коллеге (всегда одному и тому же). Докажите, что деканат может заново перераспределить студентов по трём группам, так, что в каждой отдельной группе не окажется ни одной пары студентов, из которых один даёт списывать другому. (Перераспределяйте шаг за шагом, так, чтобы каждый шаг приводил к улучшению ситуации.) [Необходимо решить задачу непременно методом потенциалов, явно указав, какая функция используется в качестве «потенциала».]
\end{task}

\begin{solution}
В качестве потенциала будем использовать число человек, которые дают списывать одногруппнику.  Если потенциал больше нуля, то выберем любого человека, который даёт списывать одногруппнику (он очевидно найдётся). Переведём его в ту группу, в которой нет человека, который даёт ему списывать. Такая найдётся, так как число групп равно трём. При таком действии потенциал уменьшается, исходя из его определения. Будем повторять это действие пока потенциал не станет равным нулю. По достижении нуля мы выполним задачу. 
\end{solution}